\begin{overview}

Competitive pressure is essential to good economic performance. But many are concerned that it is waning. They say large firms are dominating markets, pushing up prices and profits, squeezing suppliers, and slowing growth in wages and productivity. They point to the consolidation of old industries and the rise of new ones dominated by large firms.

Is competitive pressure in Australia weak? Is it waning? How costly to Australia is market power? This report assesses the evidence. It then proposes policies to increase competitive pressure. 

Large firms are not unusually dominant in Australia given the size of its economy. They do not have an unusually large share of Australian output and employment. Some large Australian markets are highly concentrated, but few are much more concentrated than in other economies Australia's size. In a modern economy, firms in many sectors have economies of scale or network effects. That is why many sectors in Australia and elsewhere are dominated by a few large firms.

The market shares of Australia's large firms have not changed much lately, on average. Their revenues have not grown faster than GDP\@. A few large sectors (such as banking) have become more concentrated, while others (such as supermarkets) have become less concentrated. In a few sectors (such as media), once-mighty firms have been disrupted by new, online competitors. Other measures of competitive pressure have not changed much either: the profitability of firms in Australia has not risen much since 2000 or become more dispersed. 

% The effect of market power on innovation is difficult to assess. Large firms, and firms in moderately concentrated sectors, innovate more than small firms. And while tech giants probably earn strong returns in Australia, they can speed the spread of good ideas and put competitive pressure on local firms. When they purchase innovative start-ups, they but  the emergence of rivals.

But competition is not uniformly strong across the Australian economy. Firms earn relatively high profits in some sectors where scale economies are strong (including supermarkets, liquor retailing, mobile phone networks, and internet service provision), in some highly regulated sectors (including banking, health insurance, and gambling), and in some some natural-monopoly sectors where competition is inherently weak (including wired telecoms, electricity distribution and transmission, and some airports). 
%
%Prices in those sectors are about 3 per cent higher, on average, than they would be if those firms earned the same profit rate as firms in sectors with lower barriers to entry. But that does not mean consumers would be better off if markets were less concentrated, because super profits in those sectors may largely reflect economies of scale. 

While profit rates suggest that there are pockets of weak competitive pressure, the economic losses are more difficult to assess because they depend on costs, not just profits. When there are just a few major firms, weak competitive pressure can permit costs to creep up, though their costs are usually still lower than those of smaller firms. Consumers probably benefit from larger firms' economies of scale. 

What should policymakers do to intensify competitive pressure? Natural-monopoly regulators need to toughen regulation. Competition regulators should continue to focus on protecting competition and preventing the misuse of market power. Governments should seek to intensify competitive pressure by reducing entry barriers, including those imposed by regulation. And they should make it easier for consumers to switch between providers and control their own data. 
\end{overview}